\chapter*{Conclusion}
\addcontentsline{toc}{chapter}{Conclusion}
\chaptermark{Conclusion}
\markboth{Conclusion}{CONCLUSION}

\numberin{equation}{chapter} 

\setcounter{chapter}{6}
\setcounter{equation}{0}


We conclude this thesis with a discussion of some of the results obtained, and indicate some future perspectives. 

Theories of nonlinear self-dual electrodynamics possess a number of interesting properties, including (i) invariance under Legendre transformation; and (ii) duality invariance the energy-momentum tensor. In chapter \ref{chap:em_duality}, we have generalized these models to nonlinear self-dual theories of a massless $\cN=1$ vector multiplet in a curved $\cN=1$ superspace. We have shown that the properties of the bosonic theories naturally generalize to the supersymmetric case, which possess (i) invariance under superfield Legendre transformation; (ii) duality invariance of the supercurrent and supertrace. These properties arise as a result of the theory satisfying the \mbox{$\cN=1$} self-duality equation (\ref{eq:sde}), which places a nontrivial restriction on the action functional. The minimal extension of the supersymmetric Born-Infeld action (\ref{eq:super BI 2}) to curved superspace emerges as a particular example of an action which does indeed satisfy this condition. 

In chapter \ref{chap:matter_coupling} we have investigated the coupling of nonlinear self-dual supersymmetric theories to matter in both the old- and new-minimal formulations of supergravity. The most interesting result concerns coupling to K{\"a}hler sigma models. The outcome of our analysis was that this coupling is nontrivial in a curved superspace, even when the chiral superfields of the nonlinear sigma model are inert under duality transformations (\ref{eq:U(1)}). In the case where the chiral superfields do transform under duality transformations, we have considered coupling to the dilaton-axion multiplet. For such models, we found that the U(1) group of duality transformation is enlarged to the non-compact group SL($2,{\mathbb R}$). We would like to point out that, to describe the dilaton-axion complex, we have used the $\cN=1$ chiral multiplet. One may also realize the dilaton-axion complex in terms of the $\cN=1$ tensor multiplet -- a situation that arises in the context of heterotic string theory. Transition from the chiral to the tensor realization can be implemented as follows. Writing the dilaton-axion K{\"a}hler potential in the form (\ref{eq:kahlerpotential}), the action (\ref{eq:dil-ax1}) can be brought (at the cost of sacrificing the manifest gauge invariance in the second line of the action) to such a form that  $\F$ and $\bar \F$ appear only in the real combination ${\rm i}(\F - \bar \F)/2$. We can then apply a superfield Legendre transformation which turns the description in terms of $\F$ and $\bar \F$ into one in terms of a {\it real} superfield ${\mathbb G}$ under the modified linearity condition
\be
({\bar \cD}^2 - 4R) {\mathbb G} = W^\a W_\a~, \qquad
(\cD^2 - 4{\bar R}) {\mathbb G} = {\bar W}_\ad {\bar W}^\ad~.
\ee
This constraint is known to describe the Chern-Simons coupling of the tensor multiplet to the vector multiplet. It would be interesting to understand what the fate of the SL$(2,{\mathbb R})$ duality symmetry is in this dual version of the theory.

In chapter \ref{chap:components} we have demonstrated the usefulness of the Kugo-Uehara method in the calculation of the component actions of the nonlinear self-dual models developed in previous chapters. The component structure of these models is extremely complicated, containing any number of derivatives. If one applies the traditional approach of component reduction to these models, the resulting action is not necessarily in a canonical form, and one must apply a field dependent Weyl and local chiral transformation (accompanied by a gravitino shift) to obtain a canonically normalized result. The Kugo-Uehara method allows us to avoid this by (i) extending the model to a super-Weyl invariant system; and (ii) when reducing to components, making a gauge choice such that the resulting action is in a canonical form. This method was implemented to derive the bosonic action of the SL($2,{\mathbb R}$) duality invariant coupling to the dilaton-axion chiral multiplet and a K{\"a}hler sigma-model (\ref{eq:bosonic action}).

With regards the fermionic dynamics investigated in chapter \ref{chap:fermionic}, we have presented three fermionic models: 
(i) $S[\j,{\bar \j}]$, the fermionic action (\ref{eq:fac}) of the vector multiplet family model (\ref{eq:family action}); 
(ii) $S[{\tilde \j},{\bar {\tilde \j}}]$, the fermionic action (equivalent to (\ref{eq:fac}) when $\j_\a={\tilde \j}_\a$) for the tensor multiplet family model (\ref{eq:tensor family}); and 
(iii) $S[\c,{\bar \c}]$, the fermionic action (\ref{eq:chiral-fermionic}) of the chiral-scalar-Goldstone action (\ref{eq:chiral-scalar-goldstone}). 
In all three cases, the actions may be brought into the Akulov-Volkov form (\ref{eq:AV}) by a nontrivial field redefinition -- for the first two actions, this field redefinition is given by (\ref{eq:field-redef}) under the condition (\ref{eq:cond-for-AV}), while for the third action it is given by (\ref{eq:field-redef2}) (correspondence checked to order $\k^4$). Now, the first two actions are the fermionic sectors of the self-dual family models (\ref{eq:family action}) and (\ref{eq:tensor family}), with $\L(\o, {\bar \o})$ a solution of the differential equation (\ref{eq:differential}). Of these families, the SBI action (\ref{eq:super BI-flat}) and the tensor-Goldstone action (\ref{eq:tensor-goldstone}) are just special representatives of the infinitely many members which satisfy the condition (\ref{eq:cond-for-AV}). It is only these special representatives (\ref{eq:super BI-flat}) and (\ref{eq:tensor-goldstone}), which describe the spontaneous partial supersymmetry breaking $\cN=2 \to \cN=1$. At the component level, however, the purely fermionic actions of all these members satisfy the condition (\ref{eq:cond-for-AV}) and are equivalent, up to a field redefinition (\ref{eq:field-redef}), to the AV action (\ref{eq:AV}), and are invariant under a second nonlinearly realized supersymmetry (\ref{eq:nonlinear-susy-comp}). In this sense, all such models can be considered to encode information about spontaneous supersymmetry breaking. 

In conclusion, we make a final comment regarding the SBI action (\ref{eq:super BI-flat}). In the purely bosonic sector, this theory reduces, upon elimination of the auxiliary field, to the Born-Infeld action
\be
\label{eq:BI}
S_{\rm BI}= \frac{1}{\k^2} \int\!\!{\rm d}^4x \,
\left(1-\sqrt{-{\rm det}(\eta_{ab}
+\k\,F_{ab})}\right)~,
\ee
{\it c.f.} (\ref{eq:BI-curved}). In the purely fermionic sector, it reduces, upon implementing the field redefinition (\ref{eq:field-redef}) with $\m=0$, to the AV action (\ref{eq:AV}). In the general case it should describe, upon implementing a nonlinear field redefinition, the spacetime filling D3-brane in a special gauge for kappa-symmetry, see \cite{Aganagic:1996nn,Tseytlin:1999dj} and references therein. Such a gauge would differ from the one chosen in \cite{Aganagic:1996nn}. It would also be of interest to obtain the full component description of the tensor-Goldstone (\ref{eq:tensor-goldstone}) and chiral-scalar-Goldstone (\ref{eq:chiral-scalar-goldstone}) theories. Such actions may be interpreted in terms of super 3-brane actions in five and six dimensional Minkowski space \cite{Rocek:1997hi}.

