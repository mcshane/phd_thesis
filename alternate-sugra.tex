\chapter{Old-minimal supergravity: alternative realization}
\label{app:alt_sugra}
Here we consider an alternative realization of the model for old-minimal supergravity  (\ref{eq:omsg2}) that is obtained by employing a variant superfield representation \cite{Gates:1980az,Gates:1980ay} of the form
\be
\S^3 = -\frac{1}{4} ({\bar \cD}^2 -4R) \,P~,
\qquad {\bar P} =P~, 
\ee
with $P$ an unconstrained real scalar superfield. It follows from (\ref{eq:SWeylTransform}) that the super-Weyl transformation of $P$ should be 
\be
\label{eq:s-weyl-P}
P ~ \to ~  {\rm e}^{- \s -{\bar \s} } \,P~,
\ee
compare with (\ref{eq:LsuperWeyl}). In this approach, the covariantly chiral scalar  $\S$ occurs not as one of the dynamical fields, but instead as a gauge-invariant field strength associated with  the following gauge freedom:
\be 
\label{eq:three-form}
\d P = L~, \qquad 
({\bar \cD}^2 - 4R ) \, L = 0~. 
\ee 
Unlike the vector multiplet case, the component vector field contained in $P$ should now be interpreted as a gauge three-form, as the action of  (\ref{eq:three-form}) on this field coincides with a three-form gauge transformation \cite{Gates:1980az,Gates:1980ay}. The gauge-invariant four-form field strength, which is associated with the three-form, appears as one of the two auxiliary fields contained in $\S^3$. The latter  implies, in fact,  that a supersymmetric cosmological term \cite{Das:1978nr,Kaku:1978nz,Kaku:1978ea,Siegel:1978nn,Siegel:1978mj} becomes real, 
\be
{\rm Im}{\int\!\!{\rm d}^8z}\, {\frac{E^{-1}}{R}}\,\S^3 =0~.
\ee
An interesting feature of this realization for old minimal supergravity is that it possesses solutions with a non-vanishing cosmological constant, without the need to explicitly add a supersymmetric cosmological term, $\m \int {\rm d}^8 z \,(ER)^{-1}\, \S^3$, to the supergravity action (\ref{eq:omsg2}). Another intriguing property of this formulation is that, unlike the standard realization, it allows a simple construction of massive off-shell supergravitons \cite{Buchbinder:2005je,Gregoire:2004ic}.

If one only  insists on the super-Weyl invariance (\ref{eq:s-weyl-P}) without requiring the gauge symmetry (\ref{eq:three-form}), a more general action, involving the naked prepotential $P$, can be considered 
\bea
\label{eq:mass}
S &=& 
{\int\!\!{\rm d}^8z}\,P\,
{\cal F} \Big( \frac{ {\bar \S} \, \S}{P} \Big)
+ \Big\{ g {\int\!\!{\rm d}^8z}\, {\frac{E^{-1}}{R}}\,
W^2 ~+~ {\rm c.c.} \Big\} ~, \\
W_\a&=& -\frac{1}{4} ({\bar \cD}^2 -4R )\cD_\a \, \ln P~,
\non
\eea
with a real function ${\cal F}(x)$ and  a constant parameter $g$. In general, such an action describes couplings of supergravity to supersymmetric matter, see e.g. \cite{deWit:1978ww}. It  corresponds to pure supergravity only if  $g=0$ and ${\cal F}(x)$ is a linear function, ${\cal F}(x) = -3x +\m$.
