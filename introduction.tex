\chapter*{Introduction}
\addcontentsline{toc}{chapter}{Introduction}
\chaptermark{Introduction}
\markboth{Introduction}{INTRODUCTION}

\vskip0.5cm

Symmetries are of great importance in all areas of physics, but few have captured the imagination and interest of theorists in quite the same way as supersymmetry. Since the pioneering papers of Gol'fand and Likhtman \cite{Golfand:1971iw}, Volkov and Akulov \cite{Volkov:1972jx,Volkov:1973ix}, and Wess and Zumino \cite{Wess:1974tw,Wess:1973kz}, much effort has been put into understanding the consequences of theories possessing this remarkable symmetry. Supersymmetry offers possible solutions to some long standing problems in particle physics, such as the gauge hierarchy problem and force unification at high energies (see, e.g., \cite{Nilles:1983ge,Haber:1984rc,Mohapatra:1986uf,Weinberg:2000cr} for reviews). It is also widely believed that a solution to the mystery of dark matter may be provided by light supersymmetric particles predicted by supersymmetric extensions of the Standard Model of particle physics (see, e.g., \cite{Bertone:2004pz} and references therein). Moreover, much impetus for current research into supersymmetry stems from its fundamental relationship with superstring theory (see \cite{Green:1987sp, Green:1987mn,Lust:1989tj,Polchinski:1998rq,Polchinski:1998rr,Zwiebach:2004tj} for reviews). Superstring theory is the only known theory with the potential to provide a unified description of the four fundamental interactions, and supersymmetry is one of the underlying principles of its formulation. However, there is currently a lack of direct experimental evidence for supersymmetry, with present particle accelerators unable to reach energies high enough to probe beyond the Standard Model. This may be addressed by the next generation accelerator being built at CERN in Geneva. The Large Hadron Collider is due to begin operating in 2007 and the detection of supersymmetry is one of its primary goals.

The defining feature of supersymmetric theories is a symmetry transforming bosons (the force mediating particles) into fermions (the matter particles) and vice versa. This symmetry, known as supersymmetry, therefore implies a (partial) unification of forces and matter. Unlike ordinary symmetries, the generators of supersymmetry transformations (supercharges) obey anticommutation relations. This implies that the classical concepts of the Lie group and Lie algebra should be properly modified in order to realize supersymmetry. The relevant algebraic structures furnishing such a realization are known as Lie supergroups and Lie superalgebras \cite{Berezin:1966nc,Berezin:1987wh,DeWitt:1992cy,Frappat:1996pb}. In four dimensions, the supersymmetric extension of the Poincar{\'e} symmetry  group (${\cal P}$), is provided by the super-Poincar{\'e} (${\cal SP}$) or $\cN$-extended super-Poincar{\'e} group. These are constructed through the addition of $\cN$ conserved Majorana spinor charges to the generators of the Poincar{\'e} algebra.

In order to keep supersymmetry manifest while performing calculations, the concepts of {\it superspace} and {\it superfields} were introduced \cite{Salam:1974yz, Ferrara:1974ac}. Superspace is an extension of spacetime by anticommuting (Grassmann) coordinates, while superfields are functions of the superspace coordinates. Unitary representations of the $\cN=1$ Poincar{\'e} superalgebra are realized in terms of superfields on a $\cN=1$ superspace. This superspace is constructed\footnote{Here ${\rm SO}(3,1)$ is the Lorentz group.} as the left coset space ${\cal SP}/{\rm SO}(3,1)$, in the same way that Minkowski space can be identified as the left coset space ${\cal P}/{\rm SO}(3,1)$. Superfields are the building blocks of supersymmetric field theories in superspace. Taking into account the anticommuting nature of the fermionic coordinates, a series expansion of a superfield in these coordinates will terminate at some order since, for all $i$ and $j$, $\theta_{i}\theta_{j}=-\theta_{j}\theta_{i}\Rightarrow\theta_{i}^{2}=0$. The coefficients in the expansion are functions of normal spacetime coordinates -- the component fields of the superfield. In this way, superfield theories can be reduced to their component fields, resulting in field theories that are supersymmetric by construction.

The introduction of gravity into the supersymmetry framework occurs quite naturally when supersymmetry transformations become local \cite{Freedman:1976xh,Deser:1976eh}. Local supersymmetry requires the inclusion of the supergravity multiplet containing the graviton and its superpartner, the gravitino. To formulate supergravity theories in superspace one can consider a curved superspace, just as Einstein introduced curved spacetime to formulate gravity. The curved superspace construction gives one a powerful tool with which to study supergravity theories in a way that is manifestly supersymmetric. In this work we will use the curved $\cN=1$ superspace formalism in order to study {\it nonlinear self-dual theories} coupled to supergravity (see the textbooks \cite{Gates:1983nr,Bagger:1990qh,Buchbinder:1998qv} and also \cite{West:1986wu,Mohapatra:1986uf,Bailin:1994qt} for comprehensive reviews of the $\cN=1$ superspace technique employed throughout).

\vskip0.5cm
%%%%%%%%%%%%%%%%%%%%%%%%%%%%%%%%%%%%%%%%%%%%%%%%
\subsection*{Nonlinear self-duality}
${}$\newline
\indent It is a well known property of Maxwell's electromagnetism that the equations of motion in vacuum\footnote{Insisting on duality invariance in the presence of matter is equivalent to the existence of particles with magnetic charges or monopoles (see, e.g., \cite{Jackson:1999xw}).} are invariant under a continuous U(1) rotation of the electric and magnetic fields, $\vec{E}+{\rm i}\vec{B}\to{\rm e}^{-{\rm i}\t}(\vec{E}+{\rm i}\vec{B})$ (see, e.g., \cite{Jackson:1999xw,Olive:1995sw}). This is equivalent to a transformation mixing the electromagnetic field strength $F_{ab}$ and its Hodge dual ${\tilde F}_{ab}=\frac{1}{2}\e_{abcd}F^{cd}$. Such transformations are called {\it electromagnetic duality rotations}. In the case $\t=\pi/2$, the transformation reads $\vec{E}\to\vec{B}$, $\vec{B}\to-\vec{E}$ (or, in terms of the field strength, $F\to{\tilde F}$, ${\tilde F}\to-F$).

In 1934, Born and Infeld \cite{Born:1934gh}, in an effort to solve the problem of the infinite self-energy of the electron, introduced a nonlinear extension of electromagnetism. It was soon noticed by Schr{\"o}dinger \cite{Schrodinger:1935} that the U(1) duality invariance of the linear theory was also present in the Born-Infeld theory, albeit in a nonlinearly realized form. In more recent times, the Born-Infeld action has been the subject of renewed interest due to its appearance in the low-energy effective actions of string theories \cite{Fradkin:1985qd,Leigh:1989jq} -- a development well beyond the original expectations that lead the authors of \cite{Born:1934gh} to put forward their theory. In conjunction with the appearance of patterns of duality invariance in extended supergravity \cite{Ferrara:1976iq,Cremmer:1979up} this led to the development by Gaillard and Zumino, Gibbons and others \cite{Gaillard:1981rj,Zumino:1981pt,Gibbons:1995cv,Gibbons:1995ap,Gaillard:1997zr,Gaillard:1997rt,Tanii:1998px,Araki:1998nn,Kimura:1999jb,Brace:1999zi,Hatsuda:1999ys,Aschieri:1999jr,Aschieri:2000dx,Ivanov:2002ab,Ivanov:2003uj} of a general theory of nonlinear self-duality in four and higher spacetime dimensions. The condition of self-duality imposes a nontrivial restriction on such theories, with the lagrangian required to satisfy a nonlinear {\it self-duality equation} (see chapter \ref{chap:em_duality}). Some interesting properties arise as a result of this self-duality equation being satisfied, including (i) invariance under Legendre transformation; and (ii) invariance of the energy momentum tensor under duality transformations. In \cite{Kuzenko:2000tg,Kuzenko:2000uh}, the general considerations of \cite{Gaillard:1981rj,Zumino:1981pt,Gibbons:1995cv,Gibbons:1995ap,Gaillard:1997zr,Gaillard:1997rt,Tanii:1998px,Araki:1998nn,Kimura:1999jb,Brace:1999zi,Hatsuda:1999ys,Aschieri:1999jr,Aschieri:2000dx} were extended to 4D $\cN=1,2$ globally supersymmetric theories and, in particular, the $\cN=1$ supersymmetric Born-Infeld theory \cite{Cecotti:1986gb} was shown to possess U(1) duality invariance. 

There exist deep yet mysterious connections between nonlinear self-duality and supersymmetry and here we give three examples. Firstly, in the case of partial spontaneous supersymmetry breakdown $\cN=2 \to \cN=1$, theories of the Maxwell-Goldstone multiplet \cite{Bagger:1996wp,Rocek:1997hi} (coinciding with the $\cN=1$ supersymmetric Born-Infeld action \cite{Cecotti:1986gb}) and the tensor-Goldstone multiplet \cite{Bagger:1997pi,Rocek:1997hi} were shown in \cite{Kuzenko:2000tg,Kuzenko:2000uh} to be self-dual, {\it i.e.} invariant under U(1) duality rotations. Additionally, the supersymmetry breaking may also be described by the chiral-scalar-Goldstone theory, which is obtained by dualizing the tensor-Goldstone model -- a procedure consistent with duality transformations. Secondly, self-duality turns out to be quite useful in attempts to construct the correct Maxwell-Goldstone multiplet action for partial supersymmetry breakdown $\cN=4 \to \cN=2$ -- the $\cN=2$ supersymmetric Born-Infeld action\footnote{See \cite{Ketov:1998sx,Ketov:2000zw} for earlier attempts to construct an $\cN=2$ supersymmetric version of the Born-Infeld action.}. It was suggested in \cite{Kuzenko:2000uh} to look for an $\cN=2$ vector multiplet action which should be (i) self-dual; and (ii) invariant under a nonlinearly realized central charge bosonic symmetry. These requirements turn out to allow one to restore the Goldstone multiplet action uniquely to any fixed order in powers of chiral superfield strength ${\cal W}$; this was carried out in \cite{Kuzenko:2000uh} up to order ${\cal W}^{10}$. Recently, there has been considerable progress in developing the formalism of nonlinear realizations to describe the partial SUSY breaking $\cN=4 \to \cN=2$ \cite{Bellucci:2000ft,Bellucci:2001hd,Bellucci:2000kc}. So far, the authors of \cite{Bellucci:2000ft,Bellucci:2001hd,Bellucci:2000kc} have reproduced the action obtained in \cite{Kuzenko:2000uh}. As a final example, we mention that the (Coulomb branch) low energy effective action of the $\cN=4$ supersymmetric Yang-Mills theory is conjectured to be invariant under U(1) duality rotations \cite{Kuzenko:2000tg} (a weaker form of self-duality of the effective action was proposed in \cite{Gonzalez-Rey:1998uh}).

The above features provide enough evidence for considering supersymmetric self-dual systems to be quite interesting and their properties worth studying. 


\vskip0.5cm
%%%%%%%%%%%%%%%%%%%%%%%%%%%%%%%%%%%%%%%%%%%%%%%%
\subsection*{Component reduction}
${}$\newline
\indent From superspace actions in terms of superfields one can reduce to a component field action. By construction, when reducing from a superfield action to components, the action will remain supersymmetric; however there is no guarantee that the component action will be in a canonically normalized form. For general supergravity-matter systems with {\it at  most two derivatives} at the component level \cite{Cremmer:1978iv,Cremmer:1978hn, Cremmer:1982wb,Cremmer:1982en}, the traditional approach (reviewed in \cite{Bagger:1990qh}) of obtaining canonically normalized component actions consists of two steps: 
(i) a plain reduction from superfields to components; and
(ii) the application of a field-dependent Weyl and local chiral transformation (accompanied by a gravitino shift).  
When we turn to models of nonlinear supersymmetric electrodynamics, a generic term in the component action may involve {\it any number of derivatives} -- with even the purely electromagnetic part of the theory being a nonlinear function of the field strength. For such supergravity-matter systems, the traditional approach can be argued to become impractical and cumbersome (as regards the component tensor calculus employed in \cite{Cremmer:1978iv,Cremmer:1978hn, Cremmer:1982wb,Cremmer:1982en}, it  has never been extended, to the best of our knowledge, to the case of the supersymmetric theories we are going to study, and therefore the superspace approach is the only formalism at our disposal). There exist two alternatives \cite{Kugo:1982mr,Binetruy:2000zx} to the traditional approach of component reduction \cite{Bagger:1990qh} that were originally developed for the systems scrutinized in \cite{Cremmer:1978iv,Cremmer:1978hn, Cremmer:1982wb,Cremmer:1982en} or slightly more generals ones, but remain equally powerful in a more general setting. The first approach, described in \cite{Binetruy:2000zx}, involves the concept of a K{\"a}hler superspace. This is a extension of conventional $\cN=1$ superspace to include a U(1) factor in the structure group. Supersymmetry coupling is then absorbed into the supergeometry by replacing the U(1) gauge potential by the superfield K{\"a}hler potential. The second approach by Kugo-Uehara \cite{Kugo:1982mr}, which we will employ in this thesis, conceptually originates in \cite{Das:1978nr,Kaku:1978nz,Kaku:1978ea} and is quite natural  in the framework of the Siegel-Gates formulation of superfield supergravity \cite{Siegel:1978nn,Siegel:1978mj}. The scheme consists of 
(i) extending the superfield theory to a super-Weyl invariant system; and then 
(ii) applying a plain component reduction along with imposing a suitable super-Weyl gauge condition. We will describe this in more detail in chapter \ref{chap:components}.


\vskip0.5cm
%%%%%%%%%%%%%%%%%%%%%%%%%%%%%%%%%%%%%%%%%%%%%%%%
\subsection*{About this thesis}
${}$\newline
\indent In this thesis we study various aspects of the supersymmetric nonlinear self-dual theories introduced in \cite{Kuzenko:2000tg,Kuzenko:2000uh}. In particular we study their properties when coupled to supergravity and supersymmetric matter. We go on to study the component structure of such theories, and in the globally supersymmetric case, investigate their relationship with the standard Akulov-Volkov (AV) action of the goldstino \cite{Volkov:1972jx,Volkov:1973ix}. The structure of the thesis is as follows:

In chapter \ref{chap:sugra} we review relevant points of the old-minimal \cite{Wess:1977fn, Grimm:1977kp, Wess:1978bu, Stelle:1978ye,Ferrara:1978em} and new-minimal \cite{Akulov:1977bu,Sohnius:1981tp} formulations of supergravity, and also review the general procedure of reducing locally supersymmetric actions from superfields to components.

Chapter \ref{chap:em_duality} begins with a brief review of nonlinear self-dual electrodynamics in curved spacetime, followed by our generalization to supersymmetric nonlinear self-dual electrodynamics in curved superspace. This involves deriving the self-duality equation as the condition for a $\cN=1$ vector multiplet model to be invariant under U(1) duality rotations. Such models are shown to be invariant under a superfield Legendre transformation. We then introduce a family of self-dual nonlinear models and argue that the supercurrent and supertrace of such models are duality invariant. 

To a large extent, the results of chapter \ref{chap:em_duality} are a minimal curved superspace extension of the globally supersymmetric results presented in \cite{Kuzenko:2000uh}. However, the considerations become more interesting when we introduce couplings to new minimal supergravity and K{\"a}hler sigma models -- such couplings are not as trivial as in the globally supersymmetric case. We investigate this in chapter \ref{chap:matter_coupling}, and also consider coupling to the dilaton-axion multiplet. Such models are of interest from the point of view of string theory.

In chapter \ref{chap:components} we commence an examination of the component structure of the models developed in chapters \ref{chap:em_duality} and \ref{chap:matter_coupling}. As will be argued in the chapter, to deal with our supergravity-matter systems, it will be useful to employ the scheme of Kugo and Uehara \cite{Kugo:1982mr}, which eliminates the need for field redefinitions and manipulations at the component level to bring the actions into canonical form. The scheme is first illustrated on the example of a nonlinear K{\"a}hler sigma-model coupled to supergravity and then used to derive the component action of new-minimal supergravity. Finally, we apply the method to investigate the component structure of the nonlinear self-dual models of chapter \ref{chap:matter_coupling}.

In chapter \ref{chap:fermionic} we turn our focus to different aspects of the fermionic dynamics of nonlinear self-dual theories, restricting ourselves to the case of global supersymmetry. We will elucidate the relationship between the fermionic sector of nonlinear self-dual models of both the vector and tensor multiplets and the AV action. The relationship is shown to be particularly special for the cases of the Maxwell-Goldstone and tensor-Goldstone models. Additionally, we look at the fermionic sector of the chiral-scalar-Goldstone model dual to the tensor-Goldstone model.

In the appendices we collect conventions (appendix \ref{app:conventions}), as well as give details of calculations not presented in the main body of the thesis, including some original results. In appendix \ref{app:alt_sugra} we present an alternative realization of old-minimal supergravity that has not been discussed in textbooks \cite{Gates:1983nr,Bagger:1990qh,Buchbinder:1998qv}. Appendix \ref{app:sd_equation} details an explicit derivation of the self-duality equation in curved superspace, while appendix \ref{app:supercurrent_supertrace} includes necessary details for the supercurrent and supertrace calculation. Additionally, some nuances of the AV action are presented in appendix \ref{app:av}. In particular, we demonstrate that all the terms of eighth order in the AV action 
completely cancel (a result which, to the best of our knowledge, has not been previously realized in the literature). Finally, details of the dualization of the tensor-Goldstone multiplet are presented in appendix \ref{app:tensor_dual}.

This thesis is based in part upon two publications \cite{Kuzenko:2002vk, Kuzenko:2005wh}. The results of \cite{Kuzenko:2002vk} are covered in chapters \ref{chap:em_duality} and \ref{chap:matter_coupling}, while those of \cite{Kuzenko:2005wh} are covered in chapters \ref{chap:components} and \ref{chap:fermionic}. The tensor multiplet and chiral scalar multiplet results presented in sections \ref{sec:tensor_multiplet} and \ref{sec:chiral_scalar} are new results that follow from \cite{Kuzenko:2005wh}, but are yet to be published.
