\begin{abstract}

\noindent Theories of self-dual supersymmetric nonlinear electrodynamics are generalized to a curved superspace of 4D $\cN = 1$ supergravity, for both the old-minimal and the new-minimal versions of $\cN = 1$ supergravity. We derive the self-duality equation, which has to be satisfied by the action functional of any U(1) duality invariant model of a massless vector multiplet, and show that such models are invariant under a superfield Legendre transformation. We construct a family of self-dual nonlinear models, which includes a minimal curved superspace extension of the $\cN = 1$ supersymmetric Born-Infeld action. The supercurrent and supertrace of such models are explicitly derived and proved to be duality invariant.

The requirement of nonlinear self-duality turns out to yield nontrivial couplings of the vector multiplet to K{\"a}hler sigma models. We explicitly construct such couplings in the case when the matter chiral multiplets are inert under the duality rotations, and more specifically to the dilaton-axion chiral multiplet when the group of duality rotations is enhanced to SL($2,{\mathbb R}$). 

The component structure of the nonlinear dynamical systems introduced proves to be more complicated, especially in the presence of supergravity, as compared with well-studied effective supersymmetric theories containing at most two derivatives (including nonlinear K{\"a}hler sigma-models). As a result, when deriving their canonically normalized component actions, the traditional approach becomes impractical and cumbersome. We find it more efficient to follow the Kugo-Uehara scheme which consists of (i) extending the superfield theory to a super-Weyl invariant system; and then (ii) applying a plain component reduction along with imposing a suitable super-Weyl gauge condition. This scheme is implemented in order to derive the bosonic action of the SL($2,{\mathbb R}$) duality invariant coupling to the dilaton-axion chiral multiplet and a K{\"a}hler sigma-model.

Another manifestation of the nontrivial component structure of nonlinear self-dual systems (of both the vector and tensor multiplets) is that their fermionic sector turns out to contain higher derivative terms, even in the case of global supersymmetry. However, we demonstrate that these higher derivative terms may be eliminated by a nontrivial field redefinition. Such a field redefinition is explicitly constructed. It brings the fermionic actions to a one-parameter deformation of the Akulov-Volkov action for the goldstino. The Akulov-Volkov form emerges, in particular, in the case of the $\cN = 1$ supersymmetric Born-Infeld action and the tensor-Goldstone multiplet action. We also analyze the fermionic sector of the chiral scalar multiplet model obtained by dualizing the tensor-Goldstone model.

\end{abstract}
