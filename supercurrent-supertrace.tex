\chapter{Calculation of the supercurrent and supertrace}
\label{app:supercurrent_supertrace}

\numberin{equation}{section} 

The calculation of the supercurrent and supertrace requires a covariant variational technique involving the disturbance of the supergravity prepotentials, {\it i.e.} a deformation of the supergeometry. We first outline the procedure as detailed in \cite{Buchbinder:1998qv}, and then go on to give some specific details for the calculation of the supercurrent and supertrace of the family model (\ref{eq:family action}).

\vskip0.5cm
%%%%%%%%%%%%%%%%%%%%%%%%%%%%%%%%%%%%%%%%%%%%%%%%
\section{Covariant variational technique}
\noindent We consider a disturbance of the supergravity prepotentials $H_a$ and $\vf\to{\rm e}^{\s}\vf$ in terms of the infinitesimal superfields ${\mathbf H}_a(z)$ and $\s(z)$, ${\bar \cD}_\ad \s=0$,
\be
{\rm e}^{-2{\rm i}{\mathbf H}^a\cD_a}\approx
1-2{\rm i}{\mathbf H}^a\cD_a~,\quad\qquad
\mbox{\boldmath $\vf$}={\rm e}^{\s}
\approx 1+\s~,
\ee
which results in a shift in the covariant derivatives as
\bea
\label{eq:prepotential_deformation}
\cD_{\a}&\longrightarrow&
\nabla_\a={\rm e}^{-{\rm i}{\mathbf H}^a\cD_a}\left(
{\cal F}\cD_\a+\frac{1}{2}\D \O_\a{}^{bc}M_{bc}
\right){\rm e}^{{\rm i}{\mathbf H}^a\cD_a}~,\\
{\bar \cD}_{\ad}&\longrightarrow&
{\bar \nabla}_\a={\rm e}^{{\rm i}{\mathbf H}^a\cD_a}\left(
{\bar {\cal F}}{\bar \cD}_\ad+
\frac{1}{2}\D {\bar \O}_\ad{}^{bc}M_{bc}
\right){\rm e}^{-{\rm i}{\mathbf H}^a\cD_a}~,\non\\
\cD_{\a\ad}&\longrightarrow&
\nabla_{\a\ad}=\frac{\rm i}{2}\{
\nabla_\a,{\bar \nabla}_\ad\}~,\non
\eea
where
\be
{\cal F}=1+\D{\cal F}~,\quad\qquad
\D{\cal F}=\frac{1}{2}\s-{\bar \s}
-\frac{1}{3}G_{a}{\mathbf H}^{a}
+\frac{2{\rm i}}{3}\cD_{a}{\mathbf H}^{a}
-\frac{1}{12}{\bar \cD}_{\ad}\cD_{\a}{\mathbf H}^{\a\ad}~,
\ee
and
\bea
\D {\bar \O}_{\ad\a\b}={\bar \cD}_{\ad}A_{\a\b}~,
&\quad&\qquad
\D {\bar \O}_{\ad\bd\gd}=
\e_{\ad\bd}{\bar \cD}_\gd \D{\bar {\cal F}}+
\e_{\ad\gd}{\bar \cD}_\bd \D{\bar {\cal F}}~,\\
A_{\a\b}&=&
\frac{1}{2}{\bar \cD}^{\bd}\cD_{(\a}{\mathbf H}_{\b)\bd}
+G_{(\a}{}^{\bd}{\mathbf H}_{\b)\bd}~.\non
\eea
Any variation (\ref{eq:prepotential_deformation}) of the supergeometry will be accompanied by a variation in the matter superfields
\be
\c \longrightarrow \c_{(\nabla)}
=f(\c\,;{\mathbf H},\mbox{\boldmath $\vf$})~,
\ee
which will depend on the superfield type. For example, a chiral scalar superfield will change as $\f_{(\nabla)}={\rm e}^{{\rm i}{\mathbf H}^a\cD_a}\f$, ${\bar \nabla}_\ad\f_{(\nabla)}=0$. The action transforms as
\be
S[\c\,;\cD]\longrightarrow S[\c_{(\nabla)}\,;\nabla]
\equiv S[\c\,;\cD\arrowvert {\mathbf H},\mbox{\boldmath $\vf$}]
={\int\!\!{\rm d}^8z}\,{\cal E}^{-1}\,
{\cal L}(\c_{(\nabla)}\,;\nabla)~.
\ee
In practice, it is helpful to switch to the so-called `quantum chiral representation',
\bea
&&{\tilde \c}_{(\nabla)}=
{\rm e}^{-{\rm i}{\mathbf H}^a \cD_a}
\c_{(\nabla)}~,\quad\qquad
{\tilde \nabla}_A=
{\rm e}^{-{\rm i}{\mathbf H}^a \cD_a}
\nabla_A
{\rm e}^{{\rm i}{\mathbf H}^a \cD_a}~,\\
&&\quad\qquad
S[\c\,;\cD\arrowvert {\mathbf H},\mbox{\boldmath $\vf$}]
={\int\!\!{\rm d}^8z}\,{\tilde {\cal E}}^{-1}\,
{\cal L}({\tilde \c}_{(\nabla)}\,;{\tilde \nabla})~,\non
\eea
where\footnote{For general calculations, expressions for how $G_{\a\ad}$ and $W_{\a\b\g}$ vary under supergravity prepotential deformations are needed. However, they were unnecessary for our calculation and we refer to \cite{Buchbinder:1998qv} for these expressions.}
\bea
\label{eq:E-Rvariation}
&&{\tilde {\cal E}}^{-1}=
E^{-1}\left(1+\s+{\bar \s}+
\frac{1}{3}G_{\a\ad}{\mathbf H}^{\a\ad}
+\frac{1}{12}[\cD_{\a},{\bar \cD}_{\ad}]
{\mathbf H}^{\a\ad}
+\frac{\rm i}{2}\cD_{\a\ad}{\mathbf H}^{\a\ad}
\right)~,\non\\
&&\phantom{E^{-1}\longrightarrow{\tilde {\cal E}}^{-1}=
E^{-1}(1}
{\tilde {\mathbf R}}=
R-\frac{1}{2}{({\bar \cD}^2 - 4 R)}\D {\bar {\cal F}}~.
\eea

To determine the supercurrent $T_a$ and supertrace $T$, ${\bar \cD}_\a T=0$, one calculates how the action has varied under (\ref{eq:prepotential_deformation}) to first order in ${\mathbf H}_a$ and $\s$,
\be
\D S=S[{\tilde \c}_{(\nabla)}\,;{\tilde \nabla}]-S[\c\,;\cD]=
{\int\!\!{\rm d}^8z}\,E^{-1}\left\{
{\mathbf H}^{a}T_{a}+\frac{1}{R}\,\s\,T
+\frac{1}{\bar R}\,{\bar \s}\,{\bar T}
\right\}~.
\ee

A useful identity, due to (\ref{eq:E-Rvariation}) and the chiral superspace rule (\ref{eq:chiralrule}), is
\be
{\int\!\!{\rm d}^8z}\,\frac{{\tilde {\cal E}}^{-1}}{{\tilde {\mathbf R}}}\,
{\cal L}_{\rm c}=
{\int\!\!{\rm d}^8z}\,\frac{E^{-1}}{R}\,(1+3\s)
{\cal L}_{\rm c}~.
\ee
We also note that there is a simple prescription for calculating the supertrace, since the local scaling of the chiral prepotential, $\vf\to{\rm e}^{\s}\vf$, with $\s(z)$ a chiral scalar superfield, ${\bar \cD}_\ad \s=0$, is just the super-Weyl transformation (\ref{eq:superweyl2}) introduced in section \ref{sec:super-weyl}. The supertrace for an action $S[\c\,;H^a,\vf]$ can then be calculated by first performing a super-Weyl transformation (\ref{eq:superweyl2}) and then taking the variational derivative with respect to $\s(z)$ and then setting $\s$ to zero,
\be
T=\frac{\d}{\d \s}
S[\c\,;H^a,{\rm e}^{\s}\vf]\Big|_{\s=0}~,
\ee
where variational differentiation with respect to a chiral scalar superfield is defined by (\ref{eq:phi-variation}).

\vskip0.5cm
%%%%%%%%%%%%%%%%%%%%%%%%%%%%%%%%%%%%%%%%%%%%%%%%
\section{Supercurrent and supertrace of the family model}
\noindent Now, we would like to use the method described above to determine the supercurrent and supertrace of the family of self-dual models (\ref{eq:family action}). 

Turning first to the supertrace, as indicated above, we first perform a super-Weyl rescaling (\ref{eq:superweyl2}) of the family action (\ref{eq:family action}). The linear part of the family action is super-Weyl invariant, while to determine the variation of the nonlinear part we use (\ref{eq:measure-super-weyl}), (\ref{eq:D2super-weyl}) and (\ref{eq:Wsuper-weyl}). We find that under an infinitesimal super-Weyl rescaling (\ref{eq:superweyl2}), the superfunctional $S[W,{\bar W}]$ varies as
\bea
\d S[W,{\bar W}]\!\!&=&\!\!-\frac{1}{2}
{\int\!\!{\rm d}^8z}\,E^{-1}\,\s\,
W^2{\bar W}^2\left(\G+{\bar \G}-\L\right)
\\ &=&
\frac{1}{8}
{\int\!\!{\rm d}^8z}\,\frac{E^{-1}}{R}\,\s\,
W^2 ({\bar \cD}^2-4R)\big[
{\bar W}^2\left(\G+{\bar \G}-\L\right)\!\big]~.
\eea
We see that this gives the supertrace $T$ (\ref{eq:supertrace}).

To determine the supercurrent, we first note that since real covariantly scalar prepotential $V$ is invariant under deformations of the supergravity prepotentials, in the quantum chiral representation we have
\be
{\tilde V}_{(\nabla)}={\rm e}^{-{\rm i}{\mathbf H}^a\cD_a}V~.
\ee
The superfield strength $W_\a$ in the quantum chiral representation is
\be
{\tilde W}_{\!(\nabla)\a}=-\frac{1}{4}
({\tilde {\bar \nabla}}^2\!\!-4{\tilde {\mathbf R}})
{\tilde \nabla}_\a {\tilde V}_{(\nabla)}
\ee
Using the identity
\be
({\tilde {\bar \nabla}}^2\!\!-4{\tilde {\mathbf R}})\l_\a=
(\d_\a{}^\b-A_\a{}^\b)({\bar \cD}^2-4R)\l_\b+({\bar \cD}^2-4R)
[A_\a{}^\b \l_\b+2\D{\bar {\cal F}}\l_\a]~,
\ee
which holds for an arbitrary spinor superfield, $\l_\a$, we can show that
\bea
{\tilde W}_{\!\!(\nabla)}^2&=&W^2
-\frac{1}{4}({\bar \cD}^2-4R)\bigg[
G_{\a\bd}{\mathbf H}^{\b\bd}(\cD^\a V)W_\b-
\frac{1}{2}({\bar \cD}_\bd \cD^\a 
{\mathbf H}^{\b\bd})(\cD_\b V)W_\a\non\\
&~&\quad\qquad
+\,{\rm i}\,{\mathbf H}^{\b\bd}(\cD_{\b\bd}\cD^\a V)W_\a-
\frac{\rm i}{2}\cD^\a({\mathbf H}^{\b\bd}\cD_{\b\bd}V)W_\a
\bigg]~.
\eea
Likewise, to determine the transformation of $\o=(1/8)({\bar \cD}^2-4R)W^2$,
\be
{\tilde {\bar \o}}_{(\nabla)}=\frac{1}{8}
({\tilde {\bar \nabla}}^2\!\!-4{\tilde {\mathbf R}})
{\tilde {\bar W}}_{\!\!(\nabla)}^2~,
\ee
we use the identity
\be
({\tilde {\bar \nabla}}^2\!\!-4{\tilde {\mathbf R}})U=
({\bar \cD}^2-4R)U+2({\bar \cD}^2-4R)
[\D{\bar {\cal F}}U]~,
\ee
which holds for arbitrary scalar superfields $U$. We also know that since $W^2$ is a covariantly chiral scalar,
\be
{\tilde {\bar W}}_{\!\!(\nabla)}^2=
{\rm e}^{-2{\rm i}{\mathbf H}^a\cD_a}
({\tilde W}_{\!\!(\nabla)}^2)^*~.
\ee

To calculate the supercurrent $T_{\a\ad}$, we must expand out
\be
\D S=S[{\tilde W}_{\!(\nabla)},
{\tilde {\bar W}}_{\!(\nabla)}
\,;{\tilde \nabla}]-S[W,{\bar W}\,;\cD]
=-\frac{1}{2}{\int\!\!{\rm d}^8z}\,
E^{-1}{\mathbf H}^{\a\ad}T_{\a\ad}~,
\ee
to first order in ${\mathbf H}_{\a\ad}$, where
\bea
S[{\tilde W}_{\!(\nabla)},
{\tilde {\bar W}}_{\!(\nabla)}
\,;{\tilde \nabla}]=
\frac{1}{4}{\int\!\!{\rm d}^8z}\,
\frac{{\tilde {\cal E}}^{-1}}
{{\tilde {\mathbf R}}}\,
{\tilde W}_{\!(\nabla)}^2\!\!&+&\!\!
\frac{1}{4}{\int\!\!{\rm d}^8z}\,
\frac{{\tilde {\cal E}}^{-1}}
{{\tilde {\bar {\mathbf R}}}}\,
{\tilde {\bar W}}_{\!(\nabla)}^2\\
\!\!&+&\!\!\frac{1}{4}{\int\!\!{\rm d}^8z}\,
{\tilde {\cal E}}^{-1}\,
{\tilde W}_{\!\!(\nabla)}^2
{\tilde {\bar W}}_{\!\!(\nabla)}^2
\L({\tilde \o}_{(\nabla)},
{\tilde {\bar \o}}_{(\nabla)})~.\non
\eea
Since $\o$ and $R$ are both chiral scalar superfields, we also have
\be
{\tilde \o}_{(\nabla)}=
{\rm e}^{-2{\rm i}{\mathbf H}^a\cD_a}
({\tilde {\bar \o}}_{(\nabla)})^*~,
\quad\qquad
{\tilde {\bar {\mathbf R}}}=
{\rm e}^{-2{\rm i}{\mathbf H}^a\cD_a}
({\tilde {\mathbf R}})^*~.
\ee
Putting all these together, after a lengthy calculation we obtain the supercurrent (\ref{eq:supercurrent}).
 

\vskip0.5cm
%%%%%%%%%%%%%%%%%%%%%%%%%%%%%%%%%%%%%%%%%%%%%%%%
\section{Duality invariance of the supercurrent and supertrace}
\noindent Duality invariance of the supercurrent and supertrace of the family model (\ref{eq:family action}) is a consequence of self-duality. To verify that the supertrace is invariant under infinitesimal duality rotations, $\d W_\a=\t M_\a$, $\d M_\a=-\t W_\a$, the following relations which result from the self-duality relation (\ref{eq:differential}) are useful
\be
\label{eq:sde-derivative}
(1-2\,{\bar \o}\,\G)\,\frac{\partial \G}{\partial \o}-
(1-2\,\o\,{\bar \G})\,\frac{\partial {\bar \G}}{\partial \o}
=-{\bar \G}^2~,\quad
(1-2\,{\bar \o}\,\G)\,\frac{\partial \G}{\partial {\bar \o}}-
(1-2\,\o\,{\bar \G})\,\frac{\partial {\bar \G}}{\partial {\bar \o}}
=\G^2~,
\ee
aswell as the reality condition
\be
\label{eq:reality}
\G-{\bar \G}+
{\bar \o}\,\frac{\partial \G}{\partial {\bar \o}}
-\o\,\frac{\partial {\bar \G}}{\partial \o}=0~.
\ee
With $M_\a$ defined by (\ref{eq:familyM}), we find that
\bea
(\d W^2){\bar W}^2&\!\!=\!\!&
-2{\rm i}\,\t\,W^2 {\bar W}^2
(1-2\,{\bar \o}\,\G)~,\\
W^2(\d {\bar W}^2)&\!\!=\!\!&
\phantom{-}2{\rm i}\,\t\,W^2 {\bar W}^2
(1-2\,\o\,{\bar \G})~,\non\\
W^2{\bar W}^2\,\d \o&\!\!=\!\!&
-2{\rm i}\,\t\,W^2 {\bar W}^2\,
\o\,(1-2\,{\bar \o}\,\G)~,\non\\
W^2{\bar W}^2\,\d {\bar \o}&\!\!=\!\!&
\phantom{-}2{\rm i}\,\t\,W^2 {\bar W}^2\,
{\bar \o}\,(1-2\,\o\,{\bar \G})~.\non
\eea
With this, if we now vary the supertrace (\ref{eq:supertrace}) directly and use (\ref{eq:sde-derivative}) and (\ref{eq:reality}), we find that $\d T=0$.

A similar calculation confirms the duality invariance of the supercurrent. When performing this check, it is easier to work with the expression
\bea
\label{eq:supercurrent2}
T_{\a \ad} ~&=&~ {\rm i} M_\a {\bar W}_\ad - {\rm i} W_\a {\bar M}_\ad
~+~ \frac{\rm i}{4} \cD_{\a \ad}\!\left(W^2 {\bar W}^2
\!\left(\G + {\bar \G} - \L\right)\right) \non \\
&-& \frac16 G_{\a \ad} W^2 {\bar W}^2 \left(\G +{\bar \G} - \L \right)
~-~ \frac{1}{24} \left[ \cD_\a , {\bar \cD}_\ad \right]
\!\left(W^2 {\bar W}^2\!\left(\G +{\bar \G} - \L \right) \right)
\non \\
&-& \frac{\rm i}{2} W^2 {\cD}_{\a \ad}\!\left[{\bar W}^2\!\left( \L
+ \frac18 {(\cD^2 - 4 {\bar R})}\!\Big( W^2 \frac{\partial \L }{\partial \o}\Big)\right)\right] \\
&+& \frac{\rm i}{2} W^2 {\bar W}^2 (\cD_{\a \ad} {\bar \o})
\frac{\partial \L }{\partial {\bar \o}}
~-~ \frac{\rm i}{16} W^2 (\cD_{\a \ad} {\bar W}^2)
{({\bar \cD}^2 - 4 R)} \Big({\bar W}^2 \, \frac{\partial \L }{\partial {\bar \o}} \Big)~,\non
\eea
which is equivalent to (\ref{eq:supercurrent}). Notice that duality invariance of the supertrace automatically implies duality invariance for the first two lines of (\ref{eq:supercurrent2}).


\numberin{equation}{chapter} 
